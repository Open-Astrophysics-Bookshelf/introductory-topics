%!TEX root = ../planets-notes.tex

These notes were written while teaching (and revising) a one-semester introductory astronomy course, ``Planets and Telescopes'' at Michigan State University during 2015, 2016, and 2017. The background required was  introductory calculus and first-year physics. The reason for the odd juxtaposition of topics is that the course was created from the merger of two undergraduate courses. During 2015 and 2016, Prof.\ Laura Chomiuk and I co-taught the course, with Laura handling the laboratory component and I the lecture. In 2017 the course became fully merged under one instructor with a graduate teaching assistant and undergraduate learning assistants helping to run the laboratory portion.

As described in the white paper describing the course revisions\cite{BrownDevelopment-of-}, the astronomy group at MSU set out learning outcomes for the undergraduate astronomy program that included an increased emphasis on analysis and interpretation of data.
In response to these goals, the revised course ``Planets and Telescopes'' uses ``real'' data and observatory archives; introduces rudimentary data analysis in both the lecture and lab components; and encourages collaborative work via a flipped classroom model.

In the first iteration of the course we used \citetalt{Lissauer2013Fundamental-Pla} and \citetalt{Bennett2013The-Cosmic-Pers} as primary texts. We found, however, that we needed to spend more time on fundamentals of astronomy, and so we subsequently switched to \citetalt{Ryden2010Foundations-of-} and \citetalt{Taylor1997An-Introduction} as primary texts and increased the amount of time spent on basics of astronomical observation and statistical analysis. In the last version of the course I taught (2017), about three weeks were spent covering the material in Appendix~\ref{ch.probability-statistics}, ``Probability and Statistics''. This was done between Chapter~\ref{ch.light-telescopes}, ``Light and Telescopes'' and Chapter~\ref{ch.detection-exoplanets}, ``Detection of Exoplanets''.  This ordering was driven by the desire to keep the lectures and labs synchronized as much as possible.

Because this is an introductory course, we often encounter students who are switching majors and need a refresher on mathematics, particularly trigonometry. I therefore added a mathematics review in Appendix~\ref{ch.math-review}.

The text layout uses the \code{tufte-book} (\url{https://tufte-latex.github.io/tufte-latex/}) \LaTeX\ class:  the main feature is a large right margin in which the students can take notes; this margin also holds small figures and sidenotes. Exercises are embedded throughout the text.  These range from ``reading exercises'' to longer, more challenging problems.  Because the exercises are embedded with the text, a list of exercises is provided in the frontmatter to help with locating material. Some of the notes and exercises on statistics are written in the form of \href{http://jupyter.org}{Jupyter Notebooks}; these are in the folder \code{statistics/notebooks}. Their usage is indicated in the text with the symbol \notebook.

I am grateful for many conversations with, and critical feedback from, Prof.~Laura Chomiuk; graduate teaching assistants Laura Shishkovsky and Alex Deibel; and undergraduate learning assistants Edward Buie III, Andrew Bundas, Claire Kopenhafer, Pham Nguyen, and Huei Sears. 

\newthought{These notes are being continuously revised;} to refer to a specific version of the notes, please use the eight-character stamp labeled ``git version'' on the front page.
