%!TEX root = ../AST208-notes.tex

These notes were written while teaching a sophomore-level astronomy course, ``Planets and Telescopes'' at Michigan State University during Spring Semester 2015. The background required is an introductory calculus sequence and a freshman-level physics course.

These notes are meant as a supplement to the course's main texts\cite{Lissauer2013Fundamental-Pla,Bennett2013The-Cosmic-Pers}.  The text layout uses the \code{tufte-book} (\url{https://tufte-latex.github.io/tufte-latex/}) \LaTeX\ class:  the main feature is a large right margin in which the students can take notes; this margin also holds small figures and sidenotes. Exercises are embedded throughout the text.  These range from ``reading exercises'' to longer, more challenging problems.  

In the course, about three weeks were spent covering the material in Appendix~\ref{ch.probability-statistics}, ``Probability and Statistics''.  This was done between covering Chapter~\ref{ch.light-telescopes}, ``Light and Telescopes'' and Chapter~\ref{ch.detection-exoplanets}, ``Detection of Exoplanets''.  This ordering was driven by the desire to keep the lectures and labs synchronized as much as possible. In Chapter~\ref{ch.coordinates}, ``Coordinates'', several of the exercises refer to the night sky as viewed from mid-Michigan in late January.

I am grateful for many conversations with, and critical feedback from,  Prof.~Laura Chomiuk, who taught the lab section of this course, graduate teaching assistant Laura Shishkovsky, and undergraduate learning assistant Andrew Bundas. 

\newthought{These notes are being continuously revised;} to refer to a specific version of the notes, please use the eight-character stamp labeled ``git version'' on the front page.
