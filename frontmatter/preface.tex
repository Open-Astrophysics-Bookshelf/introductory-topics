%!TEX root = ../planets-notes.tex

These notes were written while teaching a sophomore-level astronomy course, ``Planets and Telescopes'' at Michigan State University during Spring Semesters of 2015 and 2016. The background required is  introductory calculus and freshman-level physics. The reason for the odd juxtaposition of topics is that the course was created from the merger of two undergraduate courses, one of which was a laboratory course with observing done at the campus observatory.

In the first year, the main text was \citetalt{Lissauer2013Fundamental-Pla,Bennett2013The-Cosmic-Pers}; in the second year, we switched to \citetalt{Ryden2010Foundations-of-} and \citetalt{Taylor1997An-Introduction} and increased the amount of time spent on basics of astronomical observation and statistical analysis. Some of the notes and exercises on statistics are written in the form of \href{http://jupyter.org}{Jupyter Notebooks}; these are in the folder \code{statistics/notebooks}.

The text layout uses the \code{tufte-book} (\url{https://tufte-latex.github.io/tufte-latex/}) \LaTeX\ class:  the main feature is a large right margin in which the students can take notes; this margin also holds small figures and sidenotes. Exercises are embedded throughout the text.  These range from ``reading exercises'' to longer, more challenging problems.  Because the exercises are embedded with the text, a list of exercises is provided in the frontmatter to help with locating material.

In the course, about three weeks were spent covering the material in Appendix~\ref{ch.probability-statistics}, ``Probability and Statistics''.  This was done between covering Chapter~\ref{ch.light-telescopes}, ``Light and Telescopes'' and Chapter~\ref{ch.detection-exoplanets}, ``Detection of Exoplanets''.  This ordering was driven by the desire to keep the lectures and labs synchronized as much as possible. In Chapter~\ref{ch.coordinates}, ``Coordinates'', several of the exercises refer to the night sky as viewed from mid-Michigan in late January.

I am grateful for many conversations with, and critical feedback from,  Prof.~Laura Chomiuk, who taught the lab section of this course, graduate teaching assistants Laura Shishkovsky and Alex Deibel, and undergraduate learning assistants Edward Buie III, Andrew Bundas, Claire Kopenhafer, Pham Nguyen, and Huei Sears. 

\newthought{These notes are being continuously revised;} to refer to a specific version of the notes, please use the eight-character stamp labeled ``git version'' on the front page.
