%!TEX root = ../planets-notes.tex

These notes were written while teaching (and revising) a one-semester introductory astronomy course, ``Planets and Telescopes'' at Michigan State University during 2015, 2016, and 2017. The background required was  introductory calculus and first-year physics. The reason for the odd juxtaposition of topics is that the course was created roughly twenty years ago from the merger of two undergraduate courses.

As described in our white paper on the course development\cite{BrownDevelopment-of-}, the astronomy group at MSU committed to the following learning objectives:
\begin{quote}
Students completing an Astrophysics degree will be able to:
\begin{enumerate}
\item
Apply concepts from physics, mathematics, and scientific computing to solve quantitative problems in astrophysics;
\item
Gather, analyze, and interpret astronomical data from sources including telescopes, databases, computer simulations, analytic models, and the scientific literature; and
\item
Effectively communicate scientific ideas in both written and oral form.
\end{enumerate}
\end{quote}
As a response to these goals, our revised course ``Planets and Telescopes'' uses ``real'' data and observatory archives; introduces rudimentary data analysis in both the lecture and lab components; and encourages collaborative work via a flipped classroom model.

In the first iteration of the course we used \citetalt{Lissauer2013Fundamental-Pla} and \citetalt{Bennett2013The-Cosmic-Pers} as primary texts. We found, however, that we needed to spend more time on fundamentals of astronomy, and so we subsequently switched to \citetalt{Ryden2010Foundations-of-} and \citetalt{Taylor1997An-Introduction} as primary and increased the amount of time spent on basics of astronomical observation and statistical analysis. In the last version of the course I taught (2017), about three weeks were spent covering the material in Appendix~\ref{ch.probability-statistics}, ``Probability and Statistics''. This was done between covering Chapter~\ref{ch.light-telescopes}, ``Light and Telescopes'' and Chapter~\ref{ch.detection-exoplanets}, ``Detection of Exoplanets''.  This ordering was driven by the desire to keep the lectures and labs synchronized as much as possible. Because this was an introductory course, we did have students who were switching majors and often needed a refresher on mathematics, particularly trigonometry. I therefore added a mathematics review in Appendix~\ref{ch.math-review}.

The text layout uses the \code{tufte-book} (\url{https://tufte-latex.github.io/tufte-latex/}) \LaTeX\ class:  the main feature is a large right margin in which the students can take notes; this margin also holds small figures and sidenotes. Exercises are embedded throughout the text.  These range from ``reading exercises'' to longer, more challenging problems.  Because the exercises are embedded with the text, a list of exercises is provided in the frontmatter to help with locating material. Some of the notes and exercises on statistics are written in the form of \href{http://jupyter.org}{Jupyter Notebooks}; these are in the folder \code{statistics/notebooks}.

I am grateful for many conversations with, and critical feedback from, Prof.~Laura Chomiuk, who taught the lab section of this course; graduate teaching assistants Laura Shishkovsky and Alex Deibel; and undergraduate learning assistants Edward Buie III, Andrew Bundas, Claire Kopenhafer, Pham Nguyen, and Huei Sears. 

\newthought{These notes are being continuously revised;} to refer to a specific version of the notes, please use the eight-character stamp labeled ``git version'' on the front page.
